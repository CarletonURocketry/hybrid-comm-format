\section{Control} \label{sec:control}

\subsection{Actuation}

The pad control system is composed of several actuators which need to be controlled from the control input box at the
ground station.

All of the actuators on the system are binary actuators with only an "on" or "off" state.

Actuators will be assigned a unique numerical ID agreed upon by all nodes on the network.

All actuation requests must be responded to with an actuation acknowledgement indicating the status of the command.

\subsubsection{Actuation Request} \label{sec:act-req}

\begin{figure}[H]
    \centering
    \begin{bytefield}{16}
        \bitheader{0-15} \\
        \bitbox{8}{ID}
        \bitbox{8}{State} \\
    \end{bytefield}
    \caption{Actuation request format}
\end{figure}

\textbf{ID:} The unique numerical identifier associated with the actuator who this request is destined for.

\textbf{State:} The state to put the actuator in. Valid states are listed in Table \ref{tbl:actuator-states}.

\subsubsection{Actuation Acknowledgement} \label{sec:act-ack}

\begin{figure}[H]
    \centering
    \begin{bytefield}{16}
        \bitheader{0-15} \\
        \bitbox{8}{ID}
        \bitbox{8}{Status} \\
    \end{bytefield}
    \caption{Actuation acknowledgement format}
\end{figure}

\textbf{ID:} The unique numerical identifier associated with the actuator who this acknowledgement is from.

\textbf{Status:} The status of the request that is being acknowledged. Valid status are listed in Table
\ref{tbl:actuation-statuses}.

\begin{table}
    \centering
    \begin{tabular}{| c | c | p{4in} |}
        \hline
        \textbf{Name} & \textbf{Value} & \textbf{Description}                                                 \\
        \hline
        ACT\_OK       & 0              & The pad control system has put the actuator in the requested state.  \\
        \hline
        ACT\_DENIED   & 1              & The current arming level is too low to operate the request actuator. \\
        \hline
        ACT\_DNE      & 2              & The actuator ID is not associated with an actuator.                  \\
        \hline
        ACT\_INV      & 3              & An invalid state was requested.                                      \\
        \hline
    \end{tabular}
    \caption{Valid actuation acknowledgement statuses}
    \label{tbl:actuation-statuses}
\end{table}

\subsection{Arming}

Arming messages are used to advance the current level of arming in the pad control system. Many arming sequences must
take place remotely from the control input box at the ground station, hence their inclusion in the specification.

There are two message types for arming: the arming request and the arming acknowledgement. All arming requests are to
be responded to with an arming acknowledgement indicating the status of the request.

\subsubsection{Arming Request} \label{sec:arm-req}

\begin{figure}[H]
    \centering
    \begin{bytefield}{8}
        \bitheader{0-7} \\
        \bitbox{8}{Level} \\
    \end{bytefield}
    \caption{Arming request format}
\end{figure}

\textbf{Level:} The arming level that the client is requesting be moved to. A list of valid arming levels can be found
in Table \ref{tbl:arming-states}. Only the \texttt{ARMED\_VALVES} and the \texttt{ARMED\_IGNITION} arming levels can be
requested with this message type. The remaining arming levels are obtained by specific actuator interactions.

\subsubsection{Arming Acknowledgement} \label{sec:arm-ack}

\begin{figure}[H]
    \centering
    \begin{bytefield}{8}
        \bitheader{0-7} \\
        \bitbox{8}{Status} \\
    \end{bytefield}
    \caption{Arming acknowledgement format}
\end{figure}

\textbf{Status:} The status of the arming request which this message is acknowledging. A list of valid statuses can be
found in Table \ref{tbl:arming-statuses}.

\begin{table}
    \centering
    \begin{tabular}{| c | c | p{4in} |}
        \hline
        \textbf{Name} & \textbf{Value} & \textbf{Description}                                                                  \\
        \hline
        ARM\_OK       & 0              & The arming request was processed and the pad control system has changed arming level. \\
        \hline
        ARM\_DENIED   & 1              & The arming request could not be completed because the current arming level is
        not high enough for that level progression, or the progression must be caused via an actuator command.                 \\
        \hline
        ARM\_INV      & 2              & Invalid arming level was specified.                                                                 \\
        \hline
    \end{tabular}
    \caption{Valid arming acknowledgement statuses}
    \label{tbl:arming-statuses}
\end{table}
