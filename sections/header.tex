\section{Headers}

In order for the recipient to reliably parse incoming messages, a standardized header will precede all communication
messages. This header will specify the type of message which follows, giving the recipient enough information to parse
it. This structure will look like Figure \ref{fig:packet-structure}.

\begin{figure}[H]
    \centering
    \begin{bytefield}{16}
        \bitheader{0-15} \\
        \wordbox{1}{Header} \\
        \wordbox[lr]{1}{Message} \\
        \skippedwords \\
        \wordbox[lrb]{1}{} \\
    \end{bytefield}
    \caption{Packet structure}
    \label{fig:packet-structure}
\end{figure}

\begin{figure}[H]
    \centering
    \begin{bytefield}{16}
        \bitheader{0-15} \\
        \bitbox{8}{Type}
        \bitbox{8}{Sub-type} \\
    \end{bytefield}
    \caption{Message header}
\end{figure}

\textbf{Type:} The type of message being sent. See Table \ref{tbl:types} for valid types.

\textbf{Sub-type:} The sub type of message being sent. See Table \ref{tbl:types} for valid sub-types associated with
each type.

\begin{table}[H]
    \centering
    \begin{tabular}{| c | c | c | c | p{2in} |}
        \hline
        \textbf{Type} & \textbf{Value} & \textbf{Sub-Type} & \textbf{Value} & \textbf{Description}               \\
        \hline
        CONTROL       & 0              & ACT\_REQ          & 0              & See section \ref{sec:act-req}      \\
                      &                & ACT\_ACK          & 1              & See section \ref{sec:act-ack}      \\
                      &                & ARM\_REQ          & 2              & See section \ref{sec:arm-req}      \\
                      &                & ARM\_ACK          & 3              & See section \ref{sec:arm-ack}      \\
        \hline
        TELEMETRY     & 1              & TEMPERATURE       & 0              & See section \ref{sec:temperature}  \\
                      &                & PRESSURE          & 1              & See section \ref{sec:pressure}     \\
                      &                & MASS              & 2              & See section \ref{sec:mass}         \\
                      &                & ARMING\_STATE     & 3              & See section \ref{sec:arming-state} \\
                      &                & ACT\_STATE        & 4              & See section \ref{sec:act-state}    \\
                      &                & WARNING           & 5              & See section \ref{sec:warnings}     \\
                      &                & CONTINUITY        & 6              & See section \ref{sec:continuity} \\
        \hline
    \end{tabular}
    \caption{Valid types and sub-types}.
    \label{tbl:types}
\end{table}
